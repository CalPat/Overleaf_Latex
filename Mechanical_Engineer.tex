\documentclass[]{deedy-resume-openfont}

\begin{document}

\lastupdated

%%%%%%%%%%%%%%%%%%%%%%%%%%%%%%%%%%%%%%
%      NAME
\namesection{Calvin}{Patmont}{ 
\urlstyle{same}\url{http://calvin.patmont.com}
| \href{mailto:calvin@patmont.com}{calvin@patmont.com} 
| 925.330.3413 }


%%%%%%%%%%%%%%%%%%%%%%%%%%%%%%%%%%%%%%
%     ENG EXPERIENCE
\section{Engineering Experience}

\runsubsection{Porifera}
\descript{| Mechanical Engineer }
\datelocation{July 2015 – Current}{Hayward, CA}
Porifera is a 15 member start-up that has developed an energy efficient alternative to waste water treatment in difficult industrial and municipal settings using our patented Forward Osmosis membrane.

\begin{tightemize2}
\item Spearheaded the design and fabrication of new
Met aggressive customer time line in design and purchasing of complex water treatment skids.
%\item Designed and Fabricated automated membrane cutting station.
\item Developed pilot skids for large food \& beverage customers that met sanitary requirments.
%\item Created CAD and P\&ID drawings for FO-RO systems and automation equipment.
 \item Programmed Matlab models from empirical test data to predict system performance.
% \item Wired PLCs to communicate between sensors and high-pressure pump motors.
% \item Integrated pneumatic Motion actuated with pneumatics interfacing with PLC.
% \item Sourced parts including custom sheet-metal bent parts and explosion-proof motors to meet unique requirements.
\end{tightemize2}
\sectionsep


\runsubsection{NASA Reduced Gravity Research Program}
\descript{| Project Manager }
\datelocation{Jan 2015 – Jun 2013}{Boston University + Johnson Space Center, FL}
% The program allows aerospace engineers to demonstrate proof-of-concept of new satellite and space station parts in a zero-gravity environment.

\begin{tightemize2}
\item Designed and machined test apparatus for 2014 NASA microgravity flight.
% \item Tested satellite solar panel deployment on a 2014 NASA microgravity campaign 
\item Measured impulse force with an amplified analog signal from a Wheatstone bridge strain sensor connected through a DAQ.
%\item Learned machining techniques from trial-and-error on several parts along the way.
% \item Measured impulse force of solar panel on satellite structure with a Wheatstone bridge strain sensor connected through a DAQ.
% \item Explored fine line between over/under amplification of analog signals using Op-Amps.
% \item Developed Matlab code to solve inverse non-linear mass-spring-damper model using data collected by GoPro cameras recording solar panel deployment.
% \item Understand the importance of boundary conditions even for analyzing “good” data.
% \item Designed and fabricated a static test rig using stepper motors and a custom torsion sensor.  Data collected through an automated Matlab script that interfaces with DAQ and stepper motors drivers.
% \item Interfacing software-hardware-mechanical components never works as first intended.
\end{tightemize2}
\sectionsep


\runsubsection{University Nanosat Competition}
\descript{| Program Manager }
\datelocation{May 2013 – Mar 2012}{Boston University + AFRL (Air Force Research Laboratory)}
% Through strict deliverable schedules, design reviews and a launch opportunity, student lead teams learn first hand the rigors of spacecraft development.

\begin{tightemize2}
\item Instructed and managed 15 Engineering and Astronomy students with all disciplines of small-satellite development, and reported progress to AFRL engineers.
\item Drove requirements based design and engineering - where subsystems requirements are traced back to Science Mission and Program requirements.
% \item Instructed and managed a total of 15 Engineering and Astronomy students with all disciplines of small-satellite development, and presented progress to AFRL engineers for input.
% \item Quickly learned to breakdown roadblocks into basic components and define very specific tasks for undergrad students to complete and report weekly on.
% \item Position constantly pushed my confidence in solving complex system engineering problems often across disciplines with little prior knowledge.
% \item Formulated big-picture documents and processes such as: specifications,  CONOPS, RVM, mission success criteria, testing plans and interface coordination. Resulted in preventing student engineers to stop designing and testing until after their documentation responsibility has been completed.
\end{tightemize2}
\sectionsep

% \runsubsection{NASA VeSpR Rocket Laboratory}
% \descript{| High Vacuum Lab Tech }
% \datelocation{May 2013 – Aug 2013}{Boston University}
% \begin{tightemize2}
% \item Drafted a wiring schematic and installed low and high voltage electrical for High Vacuum chamber used for sounding rocket experiment. Sounding rocket launched in Fall 2013 from White Sands Test Center.
% \item Refurbished cryogenic pump, turbo pump, and roughing pumps in clean room for lab repair and upgrade. Fixed high-vacuum leaks.
% \item Replaced a 4-axis stepper motor stage in a large vacuum chamber with an Arduino microcontroller with code I modified and compiled.
% \end{tightemize2}
% \sectionsep



%%%%%%%%%%%%%%%%%%%%%%%%%%%%%%%%%%%%%%
%     PUBLICATION
\section{Publications}
\runpublication{\href{http://www.jossonline.com/wp-content/uploads/2016/02/Final-Design-and-Characterization-of-a-Spring-Steel-Hinge-for-Deployable-CubeSat-Structures.pdf}{Characterization of a Spring Steel Hinge for Deployable Cubesat Structures}}
\location{February 2016 | Journal of Small Satellites }
\sectionsep



\section{ARCHITECTURE \& CONSTRUCTION Experience}
\runsubsection{Hillside Homes Group}
\descript{| Project Architect }
\datelocation{Jun 2011 – Aug 2012}{Walnut Creek, CA}
% \begin{tightemize2}
% \item Designed custom homes from conceptual ideas through construction documents - three of which are successfully completed in the Bay Area.
% \item Utilized Building Information Modeling (BIM) software to accurately design and orient custom homes on difficult hillside lots.
% \item Prepared pro-forma and feasibility studies for potential land acquisitions.
% \end{tightemize2}
\sectionsep

\runsubsection{Pacific Structures}
\descript{| Construction Project Engineer }
\datelocation{Jan 2011 – Jun 2011}{San Francisco, CA}
% \begin{tightemize2}
% \item Identified and solved on-site constructability issues with the General Contractor, Architect, Mechanical, Electrical and Structural Engineers.
% \item Increased carpenters’ productivity with a prefabricated formwork system for a tedious architectural detail.
% \item Prepared material take-offs from plans and developed construction cost estimates.
% \end{tightemize2}
\sectionsep

\runsubsection{Signature Properties}
\descript{| Construction Project Engineer }
\datelocation{Jul 2007 – Jun 2010}{Oakland, CA}
% \begin{tightemize2}
% \item Project scheduling and performed subcontractor buy-out.
% \item Designed, provided structural load calcs and built unique bracing system for temporary construction elevator; estimated savings of \$50,000 and no delay.
% \end{tightemize2}
\sectionsep





%%%%%%%%%%%%%%%%%%%%%%%%%%%%%%%%%%%%%%
%     EDUCATION
\section{Education}
\runsubsection{Boston University}
\descript{| MS in Mechanical Engineering}
\location{Graduated 2015 | Boston, MA }
%\gradassistant{Graduate Researcher}{Center for Space Physics}
\location{Graduate Researcher at Center for Space Physics}
\sectionsep

\runsubsection{California Polytechnic State University}
\descript{| BS in Construction Management}
\location{Graduated 2007 | San Luis Obispo, CA}
\sectionsep

% \runsubsection{American University of Beirut}
% \descript{| Visiting Student in Architecture}
% \location{Attended 2010 | Beirut, Lebanon}
% \sectionsep





\end{document}  \documentclass[]{article}